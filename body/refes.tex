% !Mode:: "TeX:UTF-8"
% UTF-8 编辑器
\chapter{关于参考文献}

其中~chinesebst~是清泉自己所生成的参考文献风格,比较符合中国人的习惯。
\section{引用格式的修改1}

第一行在引用处数字两边加方框

第二行去除参考文献里数字两边的方框

\begin{verbatim}
\makeatletter
\def\@cite#1{\mbox{$\m@th^{\hbox{\@ove@rcfont[#1]}}$}}
\renewcommand\@biblabel[1]{#1}
\makeatother
\end{verbatim}
\section{引用格式的修改2}
\subsection*{增加~$\backslash$ucite~命令使显示的引用为上标形式}
使用下面的一种定义方法:
\begin{verbatim}
\newcommand{\ucite}[1]{$^{\mbox{\scriptsize \cite{#1}}}$}
\end{verbatim}
\begin{verbatim}
\newcommand{\ucite}[1]{\textsuperscript{\textsuperscript{\cite{#1}}}}
\end{verbatim}
\section{引用示例}
一般引用 \verb"\cite{Wang2}"
产生如:\cite{Wang2}. 包含正文中没有引用的文献可以使用\verb/\nocite{Xiedy1997}/产生~\nocite{Xiedy1997}.

上标引用 \verb"\ucite{Wang1}",产生如:\ucite{Wang1}

多个连续引用

\verb"\cite{CTug,Sangdy,Wang2,net1,net2,net3}",

产生如:\cite{CTug,Sangdy,Wang2,net1,net2,net3}.


多个上标连续引用

\verb"\ucite{CTug,Sangdy,Wang2,net1,net2,net3}",

产生如:\ucite{CTug,Sangdy,Wang2,net1,net2,net3}.
