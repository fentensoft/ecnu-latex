% !Mode:: "TeX:UTF-8"
% UTF-8 编辑器

\defaultfont


\newsavebox{\AphorismAuthor}
\newenvironment{Aphorism}[1]
{\vspace{0.5cm}\begin{sloppypar} \slshape
\sbox{\AphorismAuthor}{#1}
\begin{quote}\small\itshape }
{\\ \hspace*{\fill}------\hspace{0.2cm} \usebox{\AphorismAuthor}
\end{quote}
\end{sloppypar}\vspace{0.5cm}}


\chapter{引言}
\label{chap1}
\begin{Aphorism}{庄东辰}
在财富构成的五要素中, 健康是壹, 其余的东西都是零, 只有壹存在时, 零的增加才有意义,\cite{ren2015faster}
如果哪一天你的壹已不存在了, 你所拥有的很多零就会变成真正的零.
\end{Aphorism}
\begin{Aphorism}{ Tang Yincai. 2003.}
To help those who need the help of you will in return better your life!
\end{Aphorism}
\begin{Aphorism}{Wuyingnian. 2002.10}
Stop using Microsoft Word immediately!
\end{Aphorism}
\begin{Aphorism}{Albert Einstein}
The formulation of a problem is often more essential than its solution which may be merely a matter
of mathematical or experimental skill.
\end{Aphorism}
