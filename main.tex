%!TEX TS-program = xelatex  
%!TEX encoding = UTF-8 Unicode

\documentclass[12pt,openany,a4paper,twoside,UTF8]{ctexbook}
\usepackage{hyperref}
\hypersetup{%
%  dvipdfmx,% 设定要使用的 driver 为 dvipdfmx
  unicode={true},% 使用 unicode 来编码 PDF 字符串
  pdfstartview={FitH},% 文档初始视图为匹配宽度
  bookmarksnumbered={true},% 书签附上章节编号
  bookmarksopen={true},% 展开书签
  pdfborder={0 0 0},% 链接无框
  citecolor=blue,
  linkcolor=blue, % blue
  anchorcolor=green,
  urlcolor=blue,
  colorlinks=true,     %注释掉此项则交叉引用为彩色边框(将colorlinks和pdfborder同时注释掉)
  pdfborder=000        %注释掉此项则交叉引用为彩色边框
  %pdfstartview=FitH,
  %pdfpagemode=FullScreen % 实现打开后全屏
}
%============================ 引用的宏包 ==================================%
\input{setup/ECNU_pks.tex}

%\includeonly{body/intro}
\includeonly{preface/notation,body/intro,body/section,body/lists,body/table,body/graphs,body/equation,body/theorem,body/refes,body/conclud}

%\includeonly{body/intro,body/section,body/lists,body/table,body/graphs,body/equation,body/theorem,body/fonts,body/refes,body/conclud}

\pagestyle{fancy}
\setlength{\headheight}{20pt}
\hbadness=10000
\tolerance=10000

\begin{document}
\graphicspath{{figures/}}   %定义所有的eps文件在 figures 子目录下

%=========================== 文本格式定义 =================================%
%!TEX TS-program = xelatex  
%!TEX encoding = UTF-8 Unicode

%%%%%%%%%%%%%%%%%%%%%%%%%%%%%%%%%%%%%%%%%%%%%%%%%%%%%%%%%%%
%
% 主文档 格式定义
%
%%%%%%%%%%%%%%%%%%%%%%%%%%%%%%%%%%%%%%%%%%%%%%%%%%%%%%%%%%%


%自定义一个空命令, 用于注释掉文本中不需要的部分.
\newcommand{\comment}[1]{}

\allowdisplaybreaks
%\allowdisplaybreaks[n] %   n是1-4之间的数, 表示允许换页的程度,
                        %   比如\allowdisplaybreaks[0]表示可以换页但尽量不换,
                        %   而\allowdisplaybreaks[4]则是强制换页等同于\allowdisplaybreaks


%---------------------------- 数学公式设置 ------------------------------%
\setlength{\abovedisplayskip}{2pt plus1pt minus1pt}     %公式前的距离
\setlength{\belowdisplayskip}{2pt plus1pt minus1pt}     %公式后面的距离
\setlength{\arraycolsep}{2pt}   %在一个array中列之间的空白长度, 因为原来的太宽了


%===================================================================%
%                         各种标题样式
%===================================================================%
%======================= 标题名称中文化 ============================%
\renewcommand\contentsname{目\ 录}
\renewcommand\listfigurename{插\ 图\ 目\ 录}
\renewcommand\listtablename{表\ 格\ 目\ 录}
\renewcommand\bibname{参\ 考\ 文\ 献}    %book类型
\renewcommand\indexname{索\ 引}
\renewcommand\figurename{图}
\renewcommand\tablename{表}
%\renewcommand\partname{第\CJKnumber{\value{part}}部分}
%\renewcommand\chaptername{第\CJKnumber{\value{chapter}}章}

%=========================== 目录设置 ==================================%
\setcounter{tocdepth}{2}
\setcounter{secnumdepth}{2}

%---------------------- 定义章节的编号格式 --------------------------%
%\CTEXsetup[name={第,部分}]{part}
\CTEXsetup[name={第,章}]{chapter}
\CTEXsetup[number={\chinese{chapter}}]{chapter}
\CTEXsetup[name={\S,}]{section}
\CTEXsetup[name={\hspace{0em}\S,}]{subsection}
\CTEXsetup[name={\hspace{2\ccwd},}]{subsubsection}

%----------------------- 定义章节标题格式 ----------------------------%
%\titleformat{\appendix}[hang]{\normalfont\huge\filcenter\CJKfamily{hei}}
%    {\huge{\chaptertitlename}}{20pt}{\huge}
%\titlespacing{\appendix}{0pt}{-3ex  plus .1ex minus .2ex}{2.5ex plus .1ex minus .2ex}
%
%\titleformat{\chapter}[hang]{\normalfont\huge\filcenter\CJKfamily{hei}}
%    {\huge{\chaptertitlename}}{20pt}{\huge}
%\titlespacing{\chapter}{0pt}{-3ex  plus .1ex minus .2ex}{2.5ex plus .1ex minus .2ex}
%
%%\titleformat{\section}[hang]{\CJKfamily{hei}\Large \centering} %标题居中
%\titleformat{\section}[hang]{\CJKfamily{hei}\Large}
%    {\Large \thesection}{1em}{}{}
%\titlespacing{\section}
%    {0pt}{1.5ex plus .1ex minus .2ex}{\wordsep}
%
%\titleformat{\subsection}[hang]{\CJKfamily{hei}\large}
%    {\large \thesubsection}{1em}{}{}
%\titlespacing{\subsection}%
%    {0pt}{1.5ex plus .1ex minus .2ex}{\wordsep}
%
%\titleformat{\subsubsection}[hang]{\CJKfamily{hei}}
%    {\thesubsubsection }{1em}{}{}
%\titlespacing{\subsubsection}%
%    {0pt}{1.2ex plus .1ex minus .2ex}{\wordsep}



%====================== 定制图形和表格标题样式 =====================%
\renewcommand{\captionlabeldelim}{} %定义如  "图(表)2: 示例" 中的间隔符号,如 ":" ,这里定义为空
\renewcommand{\captionlabelsep}{\hspace{1em}} %定义图表编号与标题间的间隔距离
\renewcommand{\captionlabelfont}{\small \bf} %定义图表标签的字体
\renewcommand{\captionfont}{\small\rmfamily} %定义图表标题内容的字体

%--------------------- 定义图、表、公式的编号格式 -------------------%
%\numberwithin{equation}{section}  % used with amsmath package        %
%\numberwithin{table}{section}     % used with amsmath package        %
%\numberwithin{figure}{section}    % used with amsmath package        %
%------------------------ 定义图、表、公式的编号格式 ------------------------------%
\renewcommand{\thetable}{\arabic{chapter}-\arabic{table}}
\renewcommand{\theequation}{\arabic{chapter}-\arabic{equation}}
\renewcommand{\thefigure}{\arabic{chapter}-\arabic{figure}}

%%%@@@@@@@@@@@@@@@@@@@@@@@@@@@@@@@@@@@@@@@@@@@
%%%%%%%%%%%%%%%%%%%%%%%%%%%%%%%%%%%%%%%%%%%%%%%%%%%%%%%%
% 定义页眉和页脚 使用fancyhdr 宏包                     %
%%%%%%%%%%%%%%%%%%%%%%%%%%%%%%%%%%%%%%%%%%%%%%%%%%%%%%%%

\newcommand{\makeheadrule}{%
    \makebox[-3pt][l]{\rule[.7\baselineskip]{\headwidth}{0.4pt}}
    \rule[0.85\baselineskip]{\headwidth}{1.5pt}\vskip-.8\baselineskip}

\makeatletter
\renewcommand{\headrule}{%
    {\if@fancyplain\let\headrulewidth\plainheadrulewidth\fi
     \makeheadrule}}

%如果需要画单隔线, 则需要
\iffalse%-------------------------------%
\renewcommand{\headrulewidth}{0.5pt}    %在页眉下画一个0.5pt宽的分隔线
\renewcommand{\footrulewidth}{0pt}      % 在页脚不画分隔线.
\fi%------------------------------------%

\pagestyle{fancyplain}
\renewcommand{\chaptermark}[1]{%
\markboth{\CTEXthechapter\ #1}{}}
\renewcommand{\sectionmark}[1]{%
\markright{\S\thesection\ #1}}

\fancyhf{} %
\fancyfoot[C]{-\,\thepage\,-}
\fancyhead[LO]{\color{black}\CJKfamily{fs}\rightmark}
\fancyhead[RE]{\color{blue}\CJKfamily{fs}\leftmark}   % 在book文件类别下

%=============================== 脚注 =============================%
\renewcommand{\thefootnote}{\arabic{footnote}}



%====================================================================%
%          中文文档定理结构的设置,重定义一些正文相关标题             %
%                    针对中文稿设置                                  %
%====================================================================%

\newtheorem{definition}{\hspace{2\ccwd}{\bf{定义}}}[section]
\newtheorem{proposition}{\hspace{2\ccwd}{\bf{命题}}}[section]
\newtheorem{property}{\hspace{2\ccwd}{\bf{性质}}}[section]
\newtheorem{theorem}{\hspace{2\ccwd}{\bf{定理}}}[section]
\newtheorem{lemma}[theorem]{\hspace{2\ccwd}{\bf{引理}}}
\newtheorem{corollary}{\hspace{2\ccwd}{\bf{推论}}}  % 需要与定理一致的编号时用此命令
\newenvironment{cor}[1][\bf{推论}]{\newline\mbox{}\hspace{2\ccwd}\textbf{#1~~~}}{\hfill $\square$ \par}
\newtheorem{axiom}{\hspace{2\ccwd}{\bf{公理}}}[section]
%\newtheorem{exercise}{\hspace{2\ccwd}{\bf{习题}}}[chapter]
\newtheorem{exercise}{}[chapter]
\newtheorem{question}{\hspace{2\ccwd}{\bf{问题}}}
\newtheorem{example}{\hspace{2\ccwd}{\bf{例}}}[chapter]
%\newtheorem{exam}{\hspace{2\ccwd}例}[section]
\newtheorem{notation}{\hspace{2\ccwd}{\bf{记号}}}
\newtheorem{remark}{\hspace{2\ccwd}{\bf{注记}}}
\newtheorem{assumA}{{\bf 假设A-\hspace{-1mm}}}
\newtheorem{assumB}{{\bf 假设B-\hspace{-1mm}}}

\renewenvironment{proof}[1][证明]{\textbf{#1~~~}}{\hfill $\blacksquare$}
%\renewenvironment{proof}[1][Proof]{\textbf{#1.}}{\rule{0.5em}{0.5em}}
\newenvironment{solution}[1][解]{\textbf{#1~~~}}{\hfill $\blacksquare$} %{\hfill $\square$}
%\renewenvironment{proof}[1][Proof]{\textbf{#1.}}{\rule{0.5em}{0.5em}}


%=========================== 修改引用的格式 ==============================%

% 增加 \ucite 命令使显示的引用为上标形式
\newcommand{\ucite}[1]{\textsuperscript{\cite{#1}}}


%========================== 其它自定义 ==============================%

%====================================================================%
% 下面定义的命令(\alphtab \resettab)可以使表格编号变为 4-a, 4-b
% 使用说明:\alphtab 为开始产生处, \resettab为恢复原来表格编号形式处
% 这两个命令为自定义, 使用时应注意:不可放于 数学环境中!!!
% 在表格开始前和结束后使用!!!
%====================================================================%
%%\newcounter{savetab}%
%%\newcommand{\alphtab}{%
%%\setcounter{savetab}{\value{table}}%
%%\stepcounter{savetab}%
%%\setcounter{table}{0}%
%%%%\renewcommand{\thetable}{\arabic{savetab}-\alph{table}}%%article 中的定义
%%\renewcommand{\thetable}{\arabic{chapter}-\arabic{savetab}\alph{table}}}%%book 中的定义
%%%{\mbox{\arabic{table}-\alph{table}}}}%
%%
%%\newcommand{\resettab}{%
%%\setcounter{table}{\value{savetab}}%
%%%%\renewcommand{\thetable}{\arabic{table}}   %article 中的定义
%%\renewcommand{\thetable}{\arabic{chapter}-\arabic{table}}}  %book 中的定义


\def\defaultfont{\renewcommand{\baselinestretch}{1.5}
\fontsize{12pt}{13pt}\selectfont}



%%%%%%%%%%%%%%%%%%%%%%%%%%%%%%%%%%%%%%%%%%%%%%%%%%%%%%%%%%%%%%%%%%%%%%
% 封面、摘要、版权、致谢格式定义 --- by Feng Hua & Lei Wang
%%%%%%%%%%%%%%%%%%%%%%%%%%%%%%%%%%%%%%%%%%%%%%%%%%%%%%%%%%%%%%%%%%%%%%
\def\cdegree#1{\def\@cdegree{#1}}\def\@cdegree{}
%\def\ccovtitle#1{\def\@ccovtitle{#1}}\def\@ccovtitle{}
\def\ctitle#1{\def\@ctitle{#1}}\def\@ctitle{}
\def\caffil#1{\def\@caffil{#1}}\def\@caffil{}
\def\cmajor#1{\def\@cmajor{#1}}\def\@cmajor{}
\def\cstudy#1{\def\@cstudy{#1}}\def\@cstudy{}
\def\cauthor#1{\def\@cauthor{#1}}\def\@cauthor{}
\def\studentid#1{\def\@studentid{#1}}\def\@studentid{}
\def\csupervisor#1{\def\@csupervisor{#1}}\def\@csupervisor{}
%\def\cassosupervisor#1{\def\@cassosupervisor{~ & 副指导教师 & :& #1\\}}\def\@cassosupervisor{}
%\def\ccosupervisor#1{\def\@ccosupervisor{~ & 联\hfill 合\hfill 导\hfill 师 & :& #1\\}}\def\@ccosupervisor{}
\def\cdate#1{\def\@cdate{#1}}\def\@cdate{}
\long\def\cabstract#1{\long\def\@cabstract{#1}}\long\def\@cabstract{}
\def\ckeywords#1{\def\@ckeywords{#1}}\def\@ckeywords{}

%\def\ecovtitle#1{\def\@ecovtitle{#1}}\def\@ecovtitle{}
\def\edegree#1{\def\@edegree{#1}}\def\@edegree{}
\def\etitle#1{\def\@etitle{#1}}\def\@etitle{}
\def\eaffil#1{\def\@eaffil{#1}}\def\@eaffil{}
\def\emajor#1{\def\@emajor{#1}}\def\@emajor{}
\def\estudy#1{\def\@estudy{#1}}\def\@estudy{}
\def\eauthor#1{\def\@eauthor{#1}}\def\@eauthor{}
\def\esupervisor#1{\def\@esupervisor{#1}}\def\@esupervisor{}
\def\edate#1{\def\@edate{#1}}\def\@edate{}
\long\def\eabstract#1{\long\def\@eabstract{#1}}\long\def\@eabstract{}
\def\ekeywords#1{\def\@ekeywords{#1}}\def\@ekeywords{}

\def\makecover{
    \begin{titlepage}

% Chinese Cover

% 2013年度高校教师在职攻读硕士学位论文
    \parbox[t][4cm][c]{\textwidth}{\textbf{2013届研究生硕士学位论文}  \hfill 学校代码:~10269\hspace{3em}
    \\ \mbox{~~~~} \hspace{9.3cm}学\hphantom{校代}号:~YS00140203 }


    \parbox[t][5cm][t]{\textwidth}{
    \begin{center}
\includegraphics[height=2.2cm]{figures/ecnu_cn}
    \end{center} }

    \parbox[t][5cm][t]{\textwidth}{\Huge
    \begin{center} {\bf  \@ctitle } \end{center} }

    \parbox[t][6cm][c]{\textwidth}{ {\Large
    \begin{center}

    \renewcommand{\arraystretch}{1.0}
    \begin{tabular}{p{0cm}p{5em}l@{\extracolsep{1em}}l}
    ~ & 院\hfill 系 & & \underline{\@caffil } \\
    ~ & 专\hfill 业 & & \underline{\@cmajor}\\
    ~ & 研\hfill 究 \hfill 方 \hfill 向 & & \underline{\@cstudy}\\
    ~ & 导\hfill 师 & & \underline{\@csupervisor}\\
    ~ & 研\hfill 究 \hfill 生& & \underline{\@cauthor}\\
%    ~ & 学\hfill 号 & & \underline{\@studentid} \\
    ~ & 完\hfill 成\hfill 日\hfill 期 & & \underline{\@cdate}\\

    \end{tabular}
    \end{center} } }


    % 封二 空白页
    % English Cover
    \newpage

    \parbox[t][6cm][c]{\textwidth}{\textbf{Master Dissertation of Year 2013}  \hfill University ID:~10269\hspace{3em}
    \\ \mbox{~}\hspace{9.8cm}Student ID:~YS00140203 }

%    \parbox[t][2cm][t]{\textwidth}{
%    \begin{center}\end{center} }

    \parbox[t][4cm][t]{\textwidth}{\Huge
    \begin{center} {  \@etitle } \end{center} }

    \parbox[t][9cm][c]{\textwidth}{ {\large
    \begin{center}

    \renewcommand{\arraystretch}{1.3}
    \begin{tabular}{p{0cm}p{9em}l@{\extracolsep{1em}}l}
    ~ & Department & & \underline{\@eaffil } \\
    ~ & Major & & \underline{\@emajor}\\
    ~ & Research Direction & & \underline{\@estudy}\\
    ~ & Supervisor & & \underline{\@esupervisor}\\
    ~ & Author& & \underline{\@eauthor}\\
%    ~ & Number & & \underline{\@studentid} \\
    ~ & Date & & \underline{\@edate}\\
    \end{tabular} \end{center} } }

    \thispagestyle{empty}

    \end{titlepage}

    \normalsize

    %Authorization%%%%%%%%%%%%%%%%%%%%%%%%%%%%%%%%%%%%%%%%%%%%%%%%%%%
    \newpage

    \thispagestyle{empty}


\begin{minipage}[c]{0.95\textwidth}
{\LARGE \bf\centerline{华东师范大学学位论文原创性声明} }
\vskip0.3cm
{\normalsize\hspace{2\ccwd}郑重声明:本人呈交的学位论文《论文题目》, 是在华东师范大学攻读硕士/博士(请勾选)学位期间,
在导师的指导下进行的研究工作及取得的研究成果. 除文中已经注明引用的内容外, 本论文不包含其他个人已经发表或撰写过的研究成果.
对本文的研究做出重要贡献的个人和集体, 均已在文中作了明确说明并表示谢意. }
\end{minipage}
\parbox[t][1.5cm][c]{0.95\textwidth}{\large \hspace{3cm}
    作者签名: \hrulefill \hfill 日\hspace{2em}期: \hrulefill \hspace{1cm} }

\vskip0.7cm
\begin{minipage}[c]{0.95\textwidth}
{\LARGE \bf \centerline{华东师范大学学位论文著作权使用声明} }
\vskip0.3cm
{\normalsize\hspace{2\ccwd}《论文题目》系本人在华东师范大学攻读学位期间在导师指导下完成的硕士/博士
(请勾选)学位论文, 本论文的研究成果归华东师范大学所有. 本人同意华东师范大学根据相关规定保留和使用此学位论文,
并向主管部门和相关机构如国家图书馆、中信所和“知网”送交学位论文的印刷版和电子版;允许学位论文进入华东师范大学
图书馆及数据库被查阅、借阅;同意学校将学位论文加入全国博士、硕士学位论文共建单位数据库进行检索, 将学位论文的
标题和摘要汇编出版, 采用影印、缩印或者其它方式合理复制学位论文. }
\vskip0.3cm
\hspace{2\ccwd}本学位论文属于(请勾选)
\begin{enumerate}
\item[(\qquad)]1. 经华东师范大学相关部门审查核定的“内部”或“涉密”学位论文*,
 于\underline{\qquad}年\underline{\qquad}月\underline{\qquad}日解密, 解密后适用上述授权.
\item[(\qquad)]2. 不保密, 适用上述授权.
\end{enumerate}
\end{minipage}

\parbox[t][2.5cm][c]{0.95\textwidth}{\large \hspace{2cm}
作者签名:\hrulefill \hfill 导师签名:\hrulefill \hspace{1cm}
\\ \hspace*{2cm} 日\hspace{2em}期:\hrulefill \hfill 日\hspace{2em}期:\hrulefill \hspace{1cm} }
\vskip0.3cm\noindent
{\footnotesize * “涉密”学位论文应是已经华东师范大学学位评定委员会办公室或保密委员会审定过的学位论
文(需附获批的《华东师范大学研究生申请学位论文“涉密”审批表》方为有效),
未经上述部门审定的学位论文均为公开学位论文. 此声明栏不填写的, 默认为公开学位论文, 均适用上述授权). }

%Debating Group %%%%%%%%%%%%%%%%%%%%%%%%%%%%%%%%%%%%%%%%%%%%%%%%%%%

\newpage
\thispagestyle{empty}
\mbox{~~}
\vskip 3cm
\begin{center}
\textbf{\large XXX博士学位论文答辩委员会成员名单}
\end{center}
\begin{table}[h]
\begin{center}
\renewcommand{\arraystretch}{2.4}
  \begin{tabular}{|p{7em}|p{7em}|p{12em}|p{6em}|}
  \hline
  ~~~~姓~名  &  ~~~~职~称  &  \hspace{2em}单~~位  &  ~~~~备~注 \\ \hline
  & & & ~~~主~~席\\ \hline
  & & & \\ \hline
  & & & \\ \hline
  & & & \\ \hline
  & & & \\ \hline
  & & & \\ \hline
  & & & \\ \hline
  \end{tabular}
\end{center}
\end{table}


%Abstract and keywords%%%%%%%%%%%%%%%%%%%%%%%%%%%%%%%%%%%%%%%%%%%%%%%%%%%%%
\defaultfont

    \chapter*{摘~~~~要}
    \addcontentsline{toc}{chapter}{\bf 摘~~~~要}
    \markboth{中~文~摘~要}{中~文~摘~要}

    \@cabstract

    \vspace{1em}

    \noindent {\bf 关键词:}\@ckeywords

    \defaultfont


    \chapter*{\textsf{\textbf{Abstract}}}
    \addcontentsline{toc}{chapter}{\bf ABSTRACT(英文摘要)}
    \markboth{英~文~摘~要}{英~文~摘~要}

    \@eabstract

    \vspace{1em}

    \noindent {\textbf{Key Words:}} \quad \@ekeywords
\setcounter{page}{0}
\normalsize
\input{preface/contents}


}


\long\def\acknowledge#1{
    \defaultfont

    \chapter*{致~~~~谢}
    \markboth{致谢}{致谢}
    \addcontentsline{toc}{chapter}{\bf 致谢}

    \begin{center}
    \parbox[t][8cm][t]{\textwidth}{{\hspace{2\ccwd}#1}}
    \end{center}
}



%============================== 封面 ======================================%
\renewcommand{\baselinestretch}{1.6}    %\baselineskip 的倍数,两者相乘为行间距。
\fontsize{14.75pt}{12pt}\selectfont     %\fontsize{size}{skip}skip相当于\baselineskip

%============================= 导言部分 ===================================%
\frontmatter        %导言部分页码自动为罗马数字
\sloppy             %放松拆行的限制解决中英文混排的断行问题,会加入间距,但
                    %不会影响断行
\input{preface/cover}       %封面 (含使用授权, 中文摘要, 英文摘要)

\makecover

%%目录
%\renewcommand{\baselinestretch}{1.5}
%\fontsize{12pt}{12pt}\selectfont


%符号对照表
\include{preface/notation}

%=============================== 正文部分 ================================%
\mainmatter %进入正文页码自动变为阿拉伯数字章节计数器启动
% 对应于小四的标准字号是 12pt

%设置是正文各章使用的标准字体与行距
\renewcommand{\baselinestretch}{1.5}
\setlength{\parskip}{0.5\baselineskip}
\fontsize{12pt}{13pt}\selectfont

%正文章节 input不重新起一页,include重起一页
\setcounter{page}{1}

\include{body/intro}
\include{body/section}
\include{body/lists}
\include{body/table}
\include{body/graphs}
\include{body/equation}
\include{body/theorem}
%\include{body/fonts}
%------公式,参考文献的标签显示在页边,在论文修改时可以使用------
%\include{body/labelname}

\include{body/refes}

%建立索引的例子,需要在运行LaTeX(PDFLaTeX)后,运行 MakeIndex
%\include{body/index}


%结论
\include{body/conclud}

%附录
%----------- 定义附录中的标题、图形、表格、公式的编号格式 -------------------%
\begin{appendix}
    \renewcommand{\chaptername}{附录 \Alph{chapter}}
    \renewcommand{\thesection}{\Alph{chapter}.\arabic{section}}
    \renewcommand{\thesubsection}{\Alph{chapter}.\arabic{section}.\arabic{subsection}}
    \renewcommand{\thesubsubsection}{\arabic{subsubsection}.}
    \renewcommand{\thetable}{\Alph{chapter}-\arabic{table}}
    \renewcommand{\theequation}{\Alph{chapter}-\arabic{equation}}
    \renewcommand{\thefigure}{\Alph{chapter}-\arabic{figure}}
%   \input{appendix/A01}
%   \input{appendix/A03}
   \input{appendix/A05}
\end{appendix}


\backmatter %结束章节自动编号

%其中chinesebst是我自己生成的参考文献风格,比较符合中国人的习惯
%参考文献
\bibliographystyle{gbt7714-2005}
\phantomsection
%\bibliography{reference/reference,reference/chinese,reference/english}
\bibliography{reference/ref}

%\addcontentsline{toc}{chapter}{参考文献}
%\input{reference/refs}


%索引
\newpage
\printindex %该命令可以放到想要出现索引列表的地方
\addcontentsline{toc}{chapter}{索\ 引}

%致谢
\input{appendix/thanks}

%发表的文章列表
\input{appendix/pubs}

\clearpage
\end{document}

%%%%%%%%%%%%%%%%%% End of the file  %%%%%%%%%%%%%%%%%%%%%%%%
